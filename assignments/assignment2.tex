\documentclass[12pt]{article}
\usepackage[letterpaper,margin=1in]{geometry}
\usepackage{hyperref}
\usepackage{amsmath}
\begin{document}
\begin{center}
{\bf \LARGE AST3100 ``Code development and packaging mini-course'' Assignment 2}\\[7pt]
\emph{Due on Mar. 10 2020  at the start of class}\\[7pt]
\end{center}

The purpose of this second assignment is to start the documentation of
 the Python package that you are creating as part of this course.\\

\noindent{\bf Task 1: Write docstrings for your code} Write
numpy-style docstrings for at least a few functions, classes, and
methods in your package (remember that your package should contain a
few functions and classes and the classes should have a few methods).\\

\noindent{\bf Task 2: Set up the \texttt{sphinx} outline of your
  documentation} Make a \texttt{docs/} directory in your code package's top-level directory, enter the directory, and run\\

\texttt{sphinx-quickstart}\\

Answer the questions and then you should end up with the basic outline
of the sphinx documentation for your package. Render the HTML version
of your documentation and navigate to it in a browser to make sure
that it works.\\

\noindent{\bf Task 3: Write an installation page and a quick-start
  guide in reStructuredText} and also include them in your main
\texttt{toctree}. Render the documentation again and make sure that
all is well.\\

\noindent{\bf Task 4: Create a simple API for your code} Using the
\texttt{autodoc} extension, write an API page that list the docstrings
for all functions, classes, and their public methods.\\

\noindent{\bf Task 5: Add a simple tutorial \texttt{jupyter notebook}}
Install the \texttt{nbsphinx} extension and use it to include the
notebook as a documentation page. Render as HTML and make sure that it
works.\\

We will discuss next week how to setup \texttt{readthedocs.io} to
automatically build and share your documentation.
\end{document}
