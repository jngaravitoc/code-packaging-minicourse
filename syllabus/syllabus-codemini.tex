\documentclass{article}

\usepackage{float}
\usepackage{hyperref}
\usepackage{url}
\usepackage{amsmath,amssymb}

\include{vc}

\pagestyle{empty}

\baselineskip 18pt
\textwidth 6.25in
\textheight 8.5in
\oddsidemargin 0.1in
\evensidemargin 0.1in
\marginparwidth 0in
\marginparsep 0in
\topmargin -.5in
\topskip -.5in
\parindent 0in
\parskip 1pt

\begin{document}

\begin{center}
  \LARGE{\scshape{AST3100: Code development and packaging mini-course}}\\[5pt]
  \Large{\scshape{Winter 2020}}\\[5pt]
  \large{(last updated: \today; rev. \githash)}\\[25pt]
\end{center}

\section*{Course description}

This graduate-level mini-course provides an overview of the steps
involved in developing, publishing, and maintaining a scientific
software package. The course focuses on current best practices and the
different tools available to build, release, and maintain a software
package. The course will involve students building their own small
software package, learning how to use git/GitHub to host it, how to
test and document it, how to use automatic
documentation/building/testing tools, how to release the package, and
how to deal with issues and pull requests.

\section*{Logistics}

\begin{itemize}

  \item {\bf Meeting time / place:} Tue, 10:30--12:30pm, Room: AB
    113\\\\\emph{Sessions will typically consist of about a
      half-hour/hour of lecture/discussion and in-class work for the
      remainder of the session}

  \item {\bf Instructor:} Jo Bovy, AB 229.

  \item {\bf Email:} \href{mailto:jo.bovy@utoronto.ca}{jo.bovy@utoronto.ca}

  \item {\bf Office hours:} Stop by my office or by appointment.

  \item {\bf Course website:} \url{https://github.com/jobovy/code-packaging-minicourse}.

\end{itemize}

\section*{Learning objectives}

After this course you should understand

\begin{itemize}

  \item the dos and don'ts of package development

  \item how to setup a Python package structure that can be easily installed

  \item version control with git, hosting code on GitHub

  \item how to document your code using sphinx

  \item how to write a good test and how to run a test suite with pytest

  \item test code coverage and how to measure it

  \item code licenses

  \item how to use travis to build and test your code

  \item how to use readthedocs to host documentation
 
  \item how to release your code to PyPI (\texttt{pip install X})

  \item how to write, compile, and use a C extension in your Python code

  \item the basics of GitHub workflows

  \item how to get nifty badges for your GitHub site

\end{itemize}

\section*{Reading}

A set of lecture notes will be posted on the course website throughout
 the semester.

\section*{Grading scheme}

As part of this course, students will develop a small package in four
steps (one for each of the first four lectures); each of these steps
will be worth 25\% of the final mark.

\begin{itemize}

  \item {\bf Assignment 1:} Set up a basic package and put it on your GitHub page

  \item {\bf Assignment 2:} Set up and write documentation for your package and publish it to \texttt{readthedocs.io}.

  \item {\bf Assignment 3:} Write a test suite for your code, set it
    up to run on \texttt{travis}, send code-coverage results to
    \texttt{codecov}.

  \item {\bf Assignment 4:} Release your code to PyPI, add nifty
    badges to your GitHub page.

\end{itemize}

More information on these assignments will be given in class.

\section*{Schedule}

\begin{itemize}

  \item {\bf Week 1:} Class logistics; Introduction, git/GitHub,
    basics of Python packages, code licensing, issues and pull
    requests.

  \item {\bf Week 2:} Documentation: Sphinx, readthedocs

  \item {\bf Week 3:} Code testing, using pytest as a test runner,
    running your tests on travis, code coverage

  \item {\bf Week 4:} Releasing your code to PyPI (\texttt{pip install
    X}), get nifty badges for your GitHub site

  \item {\bf Week 5:} Advanced topics: C extensions, GitHub workflows, \ldots

\end{itemize}

\section*{Academic integrity}

From Appendix D of the Academic Integrity Handbook:
\begin{quote}
  Academic integrity is one of the cornerstones of the University of
  Toronto. It is critically important both to maintain our community
  which honours the values of honesty, trust, respect, fairness, and
  responsibility and to protect you, the students within this
  community, and the value of the degree towards which you are all
  working so diligently.  

  According to Section B of the University of
  Toronto's Code of Behaviour on Academic Matter
  (\url{http://www.governingcouncil.utoronto.ca/policies/behaveac.htm})
  which all students are expected to read and by which they are
  expected to abide, it is an offence for students to:
  \begin{itemize}
    \item Use someone else's ideas or words in their own work without
      acknowledging explicitly that those ideas/words are not their
      own with a citation and quotation marks, i.e. to commit
      plagiarism.
  \item Include false, misleading, or concocted citations in their
    work.
  \item Obtain unauthorized assistance on any assignment. 
  \item Provide unauthorized assistance to another students. This
    includes showing another student your own work.
  \item Submit their own work for credit in more than one course
      without the permission of the instructors.
  \end{itemize}

  There are other offenses covered under the Code, but these are the
  most common. You are instructed to respect these rules and the
  values that they protect.
\end{quote}

\end{document}

